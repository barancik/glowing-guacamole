\chapter{Related Work}


\section{Machine Translation Evaluation Metrics}
\todo{tohle vsechno je Marie}
An evaluation with automatic metrics has the advantages to be faster, more reproducible, and cheaper than an evaluation conducted by humans. This is especially true for the evaluation of machine translation. For a human evaluation, we would ideally need expert translators. For many language pairs, such experts are extremely rare and difficult to hire.

A large-scale and fast manual evaluation, as required by the very dynamic research area of machine translation to evaluate new systems, is often impractical. Consequently, automatic evaluation for machine translation has been a very active, and productive, research area for more than 20 years.

While BLEU remains by far the most used evaluation metric, there are countless better alternatives. BLEU is a metric from another age.

Since 2010, 100+ automatic metrics have been proposed to improve machine translation evaluation.

In this article, I present the most popular metrics that are used as alternatives, or in addition, to BLEU. I grouped them into two categories: traditional or neural metrics, each category having different advantages.
Automatic Metrics for Machine Translation

Most automatic metrics for machine translation only require:

    The translation hypothesis generated by the machine translation system to evaluate
    At least one reference translation produced by humans
    (Rarely) the source text translated by the machine translation system

Here is an example of a French-to-English translation:

    Source sentence:

Le chat dort dans la cuisine donc tu devrais cuisiner ailleurs.

    Translation hypothesis (generated by machine translation):

The cat sleeps in the kitchen so cook somewhere else.

    Reference translation:

The cat is sleeping in the kitchen, so you should cook somewhere else.

The translation hypothesis and the reference translation are both translations of the same source text.

The objective of the automatic metric is to yield a score that can be interpreted as a distance between the translation hypothesis and the reference translation. The smaller the distance is and the closer the system is to generating a translation of human quality.

The absolute score returned by a metric is usually not interpretable alone. It is almost always used to rank machine translation systems. A system with a better score is a better system.

In one of my studies (Marie et al., 2021), I showed that almost 99% of the research papers in machine translation rely on the automatic metric BLEU to evaluate translation quality and rank systems, while more than 100 other metrics have been proposed during the last 12 years. Note: I looked only at research papers published from 2010 by the ACL. Potentially many more metrics have been proposed to evaluate machine translatio
Tu vysvetlit reference, hypothesis, etc...

\subsection{Traditional}
\todo{vykradena Marie, upravit}
Traditional machine translation evaluatuin metric is based on string similarity between a reference transaltion 

Machine translation evaluation can be approached through metrics that assess the similarity between two strings based on their constituent characters. These strings consist of the translation hypothesis and the reference translation. One of the commonly used metrics in the past was WER (Word Error Rate), which was later succeeded by BLEU in the early 2000s.



Evaluation of machine translation can involve metrics that gauge the likeness between two strings, considering their component characters. These strings comprise the translation hypothesis and the reference translation. A notable historical metric is WER (Word Error Rate), which was subsequently replaced by BLEU in the early 2000s.


Advantages:

    Low Computational Cost: Traditional metrics often rely on efficient string-matching algorithms at the character and/or token level. While some metrics might involve token shifting, which could be more resource-intensive for long translations, their calculation is easily parallelizable and doesn't require GPU processing.
    Explainable: These metrics are usually straightforward to compute manually for short segments, aiding in analysis. However, this does not necessarily mean the scores are interpretable in terms of translation quality.
    Language Independence: With a few exceptions, most traditional metric algorithms can be applied across different languages without modification.

Disadvantages:

    Poor Correlation with Human Judgments: This is the main drawback compared to neural metrics. Traditional metrics do not provide a reliable estimation of translation quality.
    Specific Preprocessing Required: Except for chrF, all the traditional metrics discussed here require tokenized input segments and their reference translations. Tokenization is not integrated into the metric and must be done externally. The resulting scores are therefore reliant on specific tokenization practices that might not be consistent.

It's important to note that traditional metrics might not provide the most accurate reflection of translation quality and are being surpassed by more advanced neural metrics in this regard.


\subsubsection{With paraphrase support}


\subsection{}{Neural}




\section{Generating additional references}
\subsection{Paraphrasing}
Targeted paraphrasing for MT evaluation is introduced in~\citet{kauchak}. They 
focus on lexical substitution in Chinese-to-English translations. 
They select all pairs of words for which one word appears in a reference sentence,
second word in a hypothesis (the MT output), but none of them in both. They keep 
only pairs of synonymous words, i.e. words appearing in the same WordNet 
\cite{wordnet} synset. Each such a pair of words was further contextually evaluated. 
For every confirmed synonym, a new reference sentence is created by placing it to the
reference sentence on the position of its synonym.


%\subsection{Machine Translation}
%\cite{yoshimura_2019a} shows that metric correlation can be improved by using
%off-the-shelf MT systems (here Google Translate, Bing Translate). The 
%source sentences are translated and the translation filtered using BERT-based classifier
%trained on Microsoft Research Paraphrase Corpus (MRPC) \citep{dolan_2005}.  SentBLEU calculated using the original reference sentence + unfiltered 
%pseudo-references outperforms baseline (using  only the original reference sentence) 
%for majority of language pairs (XX->EN).
%
%\cite{albrecht_2008} use translations from off-the-shelf MT systems (Systran, 
%GoogleMT, Moses) as pseudo-references. MT systems are then evaluated directly
%(by adding the pseudo-references and computing BLEU and METEOR) and using
%newly leared metric (extracting features based on similarity with reference translation
% and monolingual fluency and training SVM to predict sentence score). This learned 
% metrics performs better than BLEU/METEOR with single references for all languages 
%but German and adding new references generally improves correlation of the standard
% evaluation metrics for most of language pairs.
% 
% Second generation metrics Meteor \cite{meteor-wmt:2014}, TERp \cite{terp} and 
%ParaEval \cite{paraeval2} still largely focus on an n-gram overlap while 
%including other linguistically motivated resources. They utilize paraphrase 
%support in form of their own paraphrase tables (i.e. collection of synonymous 
%expressions) and show higher correlation with human judgment than BLEU.
%
%Meteor supports several languages including Czech. However, its Czech 
%paraphrase tables are so noisy (i.e. they contain pairs of non-paraphrastic 
%expressions) that they actually harm the performance of the 
%metric, as it can reward mistranslated and even untranslated 
%words \cite{parmesan:2014}.
%
%String matching is hardly discriminative enough to reflect the human perception
%and there is growing number of metrics that compute their score based on rich
%linguistic features %TODO What are those?
%and matching based on parse trees, POS tagging or textual 
%entailment (e.g. \newcite{liu:2005}, \newcite{owczarzak:2007}, 
%\newcite{amigo:2009}, \newcite{Pado09}, \newcite{machacek:2011}). 
%
%These metrics shows better correlation with human judgment, but their wide 
%usage is limited by being complex and language-dependent. As a result, there is
%a trade-off between linguistic-rich strategy for better performance and 
%applicability of simple string level matching.
%
%Our approach makes use of linguistic tools for creating new reference 
%sentences. The advantage of this method is that we can choose among many 
%traditional metrics for evaluation on our new references while eliminating
%some shortcomings of these metrics. % In addition, we generate paraphrased 
%%sentences applicable not only for MT evaluation. - toto mozna uz neni nutne
%
%Targeted paraphrasing for MT evaluation was introduced by \newcite{kauchak}. 
%Their algorithm creates new reference sentences by one-word substitution 
%based on WordNet \cite{wordnet} synonymy and contextual evaluation. This 
%solution is not readily applicable to the Czech language -- a Czech word 
%has typically many forms and the correct form depends heavily on its context,
%e.g., morphological cases of nouns depend on verb valency frames. Changing a 
%single word may result in an ungrammatical sentence. Therefore, we do not 
%attempt to~change a~single word in~a~reference sentence but we focus 
%on~creating one single correct reference sentence.
%
%In \newcite{barancikova:itat2014}, we experimented with targeted
%paraphrasing using the freely available SMT system Moses \cite{moses}. 
%We adapted Moses for 
%targeted monolingual phrase-based translation. However, results of this 
%method was inconclusive. It was mainly due to a high amount of noise in the 
%translation tables and unbalanced targeting feature. %TODO explain, ale to by zabralo strasne moc prostoru. Mozna jen footnote, nebo poznamenat, ze vysvetlovani by zabralo moc prostoru a pro zajemce at se podivaj do toho clanku
%
%As a result, we rather chose to employ rule-based translation system. This 
%approach has many advantages, e.g. there is no need for creating a targeting 
%feature and we can change only parts of a sentence and thus create more 
%conservative paraphrases. We utilize Treex \cite{treex}, highly modular NLP 
%software system developed for machine translation system TectoMT \cite{tectomt} 
%that translates on a deep syntactic layer. We performed our experiment on the 
%Czech language, however, we plan to extend it to more languages, including 
%English and Spanish. 
%
%Treex is open-source and is available on GitHub,\footurl{https://github.com/ufal/treex} 
%including the two blocks that we contributed. In the rest of the paper, we describe the implementation of our approach.