\chapter*{Introduction}
\addcontentsline{toc}{chapter}{Introduction}


Mr. Praline: 'E's not pinin'! 'E's passed on! This parrot is no more! He has ceased to be! 'E's expired and gone to meet 'is maker! 'E's a stiff! Bereft of life, 'e rests in peace! If you hadn't nailed 'im to the perch 'e'd be pushing up the daisies! 'Is metabolic processes are now 'istory! 'E's off the twig! 'E's kicked the bucket, 'e's shuffled off 'is mortal coil, run down the curtain and joined the bleedin' choir invisible!! THIS IS AN EX-PARROT!! \todo{\cite{parrot}}

Above is probably the most famous paraphrases in the popular culture. It represents very well the richness/ornateness of language, how many different forms are there to express a simple fact of bird demise. 

%%Unlike artificial (programming) languages, where there is usually one correct way how to (except Perl)

This richness of a languages complicates things a bit for automatic language processing. Priklady...


Since the very first appearance of machine translation (MT) systems, a 
necessity for their objective evaluation and comparison has appeared. The 
traditional human evaluation while being the most reliable has serious 
drawbacks -- it is time-consuming and expensive.

However, the main problem of human evaluation is that it is highly dependent on 
the person annotating and there is generally low inter-annotator agreement 
\cite{wmt13}. Moreover, it is practically impossible to repeat the evaluation 
with the same results \cite{bojar-kniha}. \todo{updatovat citace}

Due to being slow and unreproducible, it is impossible to use human evaluation
for tuning and development of MT systems. Well-performing automatic MT 
evaluation metrics are essential not only for measuring the quality of translations 
and comparing different systems and approaches, but mainly for the development 
of the translation systems themselves. 


%Nowadays, automatic evaluation methods play a very important role in the development cycle of Machine Translation (MT) systems. They are used during error analysis to obtain preliminary translation quality reports over individual test cases. They are also a key ingredient in the process of system optimization, guiding the adjustment of internal system parameters. Finally, they allow for system comparison, both from the perspective of system improvement (i.e. comparison of different versions of the same system) and system competitiveness (i.e. comparison of different systems).

The pioneer metrics correlating well with human judgment were BLEU \cite{bleu} 
and NIST \citep{nist}. They are computed based on n-gram overlap between the 
MT output (hypothesis) and one or more corresponding reference sentences, i.e., 
translations made by a human translator.

Due to its simplicity and language independence, BLEU still remains de facto
standard metric for MT evaluation and tuning, even though other, 
better-performing metrics exist (\cite{wmt13-metrics}, \cite{wmt14}). 
\cite{callison2006re} shows that correlation of BLUE with human judgment is 
not as high as previously thought. Also commonly used comparing MT systems
based on their BLEU score turned out to be much rather problematic \cite{tenMattPost}

%Due to its simplicity and language independence, BLEU remains widely used, even 
%though its correlation with human judgment is not as high as previously thought
%(\cite{callison2006re},\cite{koehn-monz:2006:WMT}??). This is particular valid 
%at a sentence level (Blatz et al. 2003).

Furthermore, obtaining reference sentences is labour intensive and 
expensive,\footnote{For example, production of reference translation  at~the 
Linguistic Data Consortium is complicated process involving translation 
by~professional agencies based on elaborate guidelines and detailed quality 
control \cite{strassel}.} thus the standard practice is using only one 
reference sentence and BLEU then tends to perform badly. 

As there are many translations of a single sentence, even a perfectly correct 
hypothesis might get a low score due different wording and disregarding 
synonymous expressions (see \Fref{example_of_BLEU_malfunction}). This is 
especially valid for morphologically rich languages with free word order 
like the Czech language. \cite{bojar-tackling-sparse-data}


\begin{figure*}[tb]
\begin{center}
\begin{tabular}{ll}
 Original sentence &  \begin{tabular}{l}
  	\textit{Banks are testing payment by mobile telephone} \\
	\end{tabular}  \\
 \hline
 
 MT output & \begin{tabular}{llllll}
 			\textit{Banky} & \textit{zkou\v sej\'i} & \textit{platbu} & \textit{pomoc\'i} & \textit{mobiln\'iho} & \textit{telefonu} \\
 			Banks & are testing & payment & with help & mobile & phone \\
			\end{tabular} \\
 & \begin{tabular}{l}
  	Banks are testing payment by mobile phone \\
	\end{tabular} \\

 \hline
 Reference sentence & \begin{tabular}{llll}
 			\textit{Banky} & \textit{testuj\'i} & \textit{placen\'i} & \textit{mobilem} \\
 			Banks & are testing & paying & by mobile phone \\
			\end{tabular} \\
 &  \begin{tabular}{l}
  	Banks are testing paying by mobile phone \\
	\end{tabular}
 
\end{tabular}
\caption{Example from WMT12 - Even though the translation is grammatically 
correct and the meaning of both sentences is very similar, it doesn't contribute 
to the BLEU score. There is only one unigram overlapping.}
\end{center}
\label{example_of_BLEU_malfunction}
\end{figure*}

As the main task of an MT metric is essentially to decide whether a reference
sentence is a paraphrase of a given hypothesis, our goal is to achieve higher 
accuracy of MT evaluation by targeted paraphrasing of reference sentences., 
i.e. creating a new synthetic reference sentence that is still correct and 
keeps the meaning of the original sentence but at the same time it is closer in 
wording to the MT output. BLEU and other string-based metrics performs more 
reliable using these new references. %(TODO: nejak preformulovat)

The structure of the thesis is as follows: in the next section, we present 
other work in paraphrasing for MT evaluation. In \Sref{Data}, we introduced
the data we use for our experiments. In sections \ref{lrec} and \ref{MT}, 
we show paraphrasing based on phrase substitutions and machine translation 
itself. Finally, we conclude with avenues for further work.





BLEU \cite{bleu} remains the most common metric for MT evaluation, even
though other, better-performing metrics exist \cite{wmt13-metrics}. BLEU is 
computed from the number of phrase overlaps between the translated sentence and 
the corresponding reference sentences, i.e., translations made by a professional 
human translator. 

The advantage of BLEU is its simplicity and language independence. But it performs 
very badly for morphologically rich languages %with free word order 
like the Czech language \cite{bojar-tackling-sparse-data}, especially when only a 
single reference sentence is used.

One of the reasons is that BLEU %(and other commonly used metrics - NIST \cite{nist}, 
%TER \cite{ter} etc.) 
disregards synonymous phrases and word form variants. One way 
how to  alleviate this drawback is to include paraphrases to the evaluation metric 
(e.g. METEOR \cite{meteor}). 

But this often awards even sentences with paraphrases that are not grammatically correct. 
We take a different approach by transforming a reference translation into a sentence that 
is closer to the MT output and keeps its original meaning and correctness.



------------


%, although there were recently successful attempts to overcome this problem 
%using crowdsourcing techniques. [\markcite{{\it Callison-Burch}, 2009]} and 
%[\markcite{{\it Bentivogli et al.}, 2011]} demonstrate that the average score of 
%high number of non-expert low-paid annotators shows high agreement with gold 
%standards provided by expert annotators in various natural language processing 
%(NLP) tasks including machine translation evaluation.

%Pioneer measures such as BLEU and NIST measure MT quality cheaply and objectively
%through the strong correlation between human judgement and the n-gram overlap
%betweeen  aa system translation and one or more reference translation. 

%The first metric correlating well with human judgment was BLEU \cite{bleu}.
%Due to its simplicity and language-independence, it still remains the most 
%common metric for MT evaluation, even though other, better-performing metrics 
%exist. \cite{wmt14} 

